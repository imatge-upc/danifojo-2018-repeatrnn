%!TEX root = ../main.tex
% Chapter Template

\chapter{Experiments} % Main chapter title

\label{Chapter5} % Change X to a consecutive number; for referencing this chapter elsewhere, use \ref{ChapterX}

%----------------------------------------------------------------------------------------
%	SECTION 1
%----------------------------------------------------------------------------------------
All experiments are run with ADAM optimizer and learning rate of \(10^{-4}\)
\section{Addition}

The addition task presents the network with a input sequence of 1 to 5 size 50 input vectors. Each vector represents a \(D\) digit number, where \(D\) is drawn randomly from 1 to 5, and each digit is drawn randomly from 0 to 9. The first \(10D\) elements of the vector are a concatenation of one-hot encodings of the \(D\) digits in the number,  and the remainder of the vector is set to 0.  The required output is the cumulative sum of all inputs up to the current one,  represented as a set of 6 simultaneous classifications  for  the  6  possible  digits  in  the  sum. There  is  no  target  for  the  first  vector  in  the  sequence, as no sums have yet been calculated.  Because the previous sum must be carried over by the network,  this task again requires the internal state of the network to remain coherent.  Each classification is modelled by a size 11 softmax, where the first 10 classes are the digits and the \(11^{\text{th}}\) is a special marker used to indicate that the number is complete.

The  network  was  single-layer  LSTM  with  512  memory  cells. The  loss  function  was  the  joint cross-entropy of all 6 targets at each time-step where targets were present and the minibatch size was 32. The maximum ponder \(M\) was set to 20 for this task, as it was found that some networks had very high ponder times early in training

%-----------------------------------
%	SUBSECTION 1
%-----------------------------------
\subsection{Subsection 1}

Nunc posuere quam at lectus tristique eu ultrices augue venenatis. Vestibulum ante ipsum primis in faucibus orci luctus et ultrices posuere cubilia Curae; Aliquam erat volutpat. Vivamus sodales tortor eget quam adipiscing in vulputate ante ullamcorper. Sed eros ante, lacinia et sollicitudin et, aliquam sit amet augue. In hac habitasse platea dictumst.

%-----------------------------------
%	SUBSECTION 2
%-----------------------------------

\subsection{Subsection 2}
Morbi rutrum odio eget arcu adipiscing sodales. Aenean et purus a est pulvinar pellentesque. Cras in elit neque, quis varius elit. Phasellus fringilla, nibh eu tempus venenatis, dolor elit posuere quam, quis adipiscing urna leo nec orci. Sed nec nulla auctor odio aliquet consequat. Ut nec nulla in ante ullamcorper aliquam at sed dolor. Phasellus fermentum magna in augue gravida cursus. Cras sed pretium lorem. Pellentesque eget ornare odio. Proin accumsan, massa viverra cursus pharetra, ipsum nisi lobortis velit, a malesuada dolor lorem eu neque.

%----------------------------------------------------------------------------------------
%	SECTION 2
%----------------------------------------------------------------------------------------

\section{Parity}

Sed ullamcorper quam eu nisl interdum at interdum enim egestas. Aliquam placerat justo sed lectus lobortis ut porta nisl porttitor. Vestibulum mi dolor, lacinia molestie gravida at, tempus vitae ligula. Donec eget quam sapien, in viverra eros. Donec pellentesque justo a massa fringilla non vestibulum metus vestibulum. Vestibulum in orci quis felis tempor lacinia. Vivamus ornare ultrices facilisis. Ut hendrerit volutpat vulputate. Morbi condimentum venenatis augue, id porta ipsum vulputate in. Curabitur luctus tempus justo. Vestibulum risus lectus, adipiscing nec condimentum quis, condimentum nec nisl. Aliquam dictum sagittis velit sed iaculis. Morbi tristique augue sit amet nulla pulvinar id facilisis ligula mollis. Nam elit libero, tincidunt ut aliquam at, molestie in quam. Aenean rhoncus vehicula hendrerit.